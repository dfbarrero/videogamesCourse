%%%%%%%%%%%%%%%%%%%%%%%%%%%%%%%%%%%%%%%%%%%%%%%%%%%%%%%%%%%%%%%%%
% MUW Presentation
% LaTeX Template
% Version 1.0 (27/12/2016)
%
% License:
% CC BY-NC-SA 4.0 (http://creativecommons.org/licenses/by-nc-sa/3.0/)
%
% Created by:
% Nicolas Ballarini, CeMSIIS, Medical University of Vienna
% nicoballarini@gmail.com
% http://statistics.msi.meduniwien.ac.at/
%
% Customized for UAH by:
% David F. Barrero, Departamento de Automática, UAH
%%%%%%%%%%%%%%%%%%%%%%%%%%%%%%%%%%%%%%%%%%%%%%%%%%%%%%%%%%%%%%%%%

\documentclass[10pt,compress]{beamer} % Change 10pt to make fonts of a different size
\mode<presentation>

\usepackage[spanish]{babel}
\usepackage{fontspec}
\usepackage{tikz}
\usepackage{etoolbox}
\usepackage{xcolor}
\usepackage{xstring}
\usepackage{listings}

% Introduced by David
\usepackage{eurosym}

\usetheme{UAH}
\usecolortheme{UAH}
\setbeamertemplate{navigation symbols}{} 
\setbeamertemplate{caption}[numbered]

%%%%%%%%%%%%%%%%%%%%%%%%%%%%%%%%%%%%%%%%%%%%%%%%%%%%%%%%%%%%%%%%%
%% Presentation Info
\title[Videogame delivery notes]{Videogame project delivery notes}
\author{\asignatura\\\carrera}
\institute{}
\date{Departamento de Automática}
%%%%%%%%%%%%%%%%%%%%%%%%%%%%%%%%%%%%%%%%%%%%%%%%%%%%%%%%%%%%%%%%%


%%%%%%%%%%%%%%%%%%%%%%%%%%%%%%%%%%%%%%%%%%%%%%%%%%%%%%%%%%%%%%%%%
%% Descomentar para habilitar barra de navegación superior
%\setNavigation
%%%%%%%%%%%%%%%%%%%%%%%%%%%%%%%%%%%%%%%%%%%%%%%%%%%%%%%%%%%%%%%%%

%%%%%%%%%%%%%%%%%%%%%%%%%%%%%%%%%%%%%%%%%%%%%%%%%%%%%%%%%%%%%%%%%
%% Configuración de logotipos en portada
%% Opacidad de los logotipos
\newcommand{\opacidad}{1}
%% Descomentar para habilitar logotipo en pié de página de portada
\renewcommand{\logoUno}{Images/isg.png}
%% Descomentar para habilitar logotipo en pié de página de portada
%\renewcommand{\logoDos}{Images/CCLogo.png}
%% Descomentar para habilitar logotipo en pié de página de portada
%\renewcommand{\logoTres}{Images/ALogo.png}
%% Descomentar para habilitar logotipo en pié de página de portada
%\renewcommand{\logoCuatro}{Images/ELogo.png}
%%%%%%%%%%%%%%%%%%%%%%%%%%%%%%%%%%%%%%%%%%%%%%%%%%%%%%%%%%%%%%%%%

%%%%%%%%%%%%%%%%%%%%%%%%%%%%%%%%%%%%%%%%%%%%%%%%%%%%%%%%%%%%%%%%%
%% FOOTLINE
%% Comment/Uncomment the following blocks to modify the footline
%% content in the body slides. 


%% Option A: Title and institute
\footlineA
%% Option B: Author and institute
%\footlineB
%% Option C: Title, Author and institute
%\footlineC
%%%%%%%%%%%%%%%%%%%%%%%%%%%%%%%%%%%%%%%%%%%%%%%%%%%%%%%%%%%%%%%%%

\begin{document}

%%%%%%%%%%%%%%%%%%%%%%%%%%%%%%%%%%%%%%%%%%%%%%%%%%%%%%%%%%%%%%%%%
% Use this block for a blue title slide with modified footline
{\titlepageBlue
    \begin{frame}
        \titlepage
    \end{frame}
}

\begin{frame}[plain]{}
   \begin{block}{Objectives}
   \begin{itemize}
        \item Clarify how to deliver the videogame project
		\item Point out some practical issues
	\end{itemize}
	\end{block}

   \begin{block}{Bibliography}
   		None
   \end{block}
\end{frame}

%{
%\disableNavigation{white}
%\begin{frame}[shrink]{Table of Contents}
% \frametitle{Table of Contents}
% \tableofcontents
  % You might wish to add the option [pausesections]
%\end{frame}
%}


\begin{frame}{Project deliverables}
	\begin{block}{Project deliverables}
	\begin{itemize}
	\item Videogame source code
	\item Design document: How the game will be
	\item Project report: What has been done
	\end{itemize}
	\end{block}

	\bigskip
	
	Delivery method $\rightarrow$ GitHub
	\begin{itemize}
	\item Source code in the repository
	\item Documentation also in the repository
		%\begin{itemize}
		%\item Host the web site in GitHub
		%\item You are allowed to use any technology ...
		%\item ... however I recommend Jekyll
		%\end{itemize}
	\end{itemize}
\end{frame}

\begin{frame}{Evaluation procedure}
	After the examan, the instructor will
	\begin{enumerate}
	\item Fork all the repositories
	\item Download the project from GitHub
	\item Try to execute the project
	\item Read the documentatoin and source code
	\item Observe GitHub activity
	\item Set individual and group califications
	\end{enumerate}

	Evaluation platform:
	\begin{itemize}
	\item PC with Ubuntu
	\end{itemize}
\end{frame}

\begin{frame}{Evaluation criteria}
	\begin{itemize}	
	\item Technical quality (group) - 20\%
	    \begin{itemize}
	    \item $0$ if the videogame does not properly execute
	    \end{itemize}
	\item Game design (group) - 20\%
	    \begin{itemize}
	    \item Game originality will have a big impact in this criteria
	    \end{itemize}
	\item Documentation quality (group) - 10\%
	\item Teamwork (individual) - 20\%
	    \begin{itemize}
	    \item Activity in GitHub and coevaluatoin will have a big impact in this criteria
	    \end{itemize}
	\end{itemize}
\end{frame}

\begin{frame}{Source code structure}
	Recommended source code structure
	\begin{itemize}
	\item \texttt{src/} or \texttt{<project name>}: Source code
	\item \texttt{docs/}: Documentation
		\begin{itemize}
		\item \texttt{docs/gdd.md}: Game Design Document
		\item \texttt{docs/memoria.md}: Project memory
		\item \texttt{docs/minutas.md}: Meetings minutes
		\end{itemize}
	\item \texttt{tests/}: Unitary tests
	\item \texttt{dist/}: Distribution
	\item \texttt{assets/}:
		\begin{itemize}
		\item \texttt{assets/sprites}:
		\item \texttt{assets/maps}
		\item \texttt{assets/music}
		\item \texttt{assets/sound}
		\end{itemize}
	\end{itemize}
   In general, always try to keep your code neat and nice
\end{frame}

%\begin{frame}{Memory contents}
%	\begin{enumerate}
%	\item Team members and roles. Do not include missing people
%	\item How the team has been organized
%	\item What degree of the GDD has been accomplished
%	\item Explain why the GDD has not been fully accomplished, if it applies
 %   \item Technical aspects of the project you want to be considered
  %  \item Any additional information that the instructor should know
%	\end{enumerate}
 %   Remember CCC: context, content, conclussions
%\end{frame}

\begin{frame}[fragile]{Typical problems (I)}
    \begin{columns}
 	   \column{.50\textwidth}
		\begin{alertblock}{Paths}
		Use relative paths, ALWAYS
		\end{alertblock}

   	 	\column{.50\textwidth}
		\begin{alertblock}{Path separator}
		Use proper path separator
		\end{alertblock}
   	\end{columns}

	\bigskip

	Windows style:
	\begin{itemize}
	\item \texttt{resources\textbackslash sprites\textbackslash alien.png}
	\end{itemize}

	Unix style (Linux and MacOS):
	\begin{itemize}
	\item \texttt{resources/sprites/alien.png}
	\end{itemize}

	Python trick: \texttt{os.path.sep}
	\begin{itemize}
		\item \texttt{spritePath = 'resources' + os.path.sep + 'sprites'}
	\end{itemize}
\end{frame}

\begin{frame}[fragile]{Typical problems (II)}
    Extra trick:
    \begin{itemize}
        \item Any relative path depends on the working directory
        \item Solution: change working directory
    \end{itemize}

    \begin{block}{}
    \begin{verbatim}
file_path = os.path.dirname(os.path.abspath(__file__))
os.chdir(file_path)
\end{verbatim}
\end{block}

    Not everybody has a RTX 3090TI ...
    \begin{itemize}
        \item Use a reasonable resolution
    \end{itemize}
\end{frame}

\begin{frame}{Last remarks}
	Remember to test the game properly ...
	    \begin{itemize}
		    \item ... and ideally using an issue tracker
	    \end{itemize}
	\bigskip
	Python app deployment: \url{https://wiki.python.org/moin/deployment}
\end{frame}

\end{document}
