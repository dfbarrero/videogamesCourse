%%%%%%%%%%%%%%%%%%%%%%%%%%%%%%%%%%%%%%%%%%%%%%%%%%%%%%%%%%%%%%%%
% MUW Presentation
% LaTeX Template
% Version 1.0 (27/12/2016)
%
% License:
% CC BY-NC-SA 4.0 (http://creativecommons.org/licenses/by-nc-sa/3.0/)
%
% Created by:
% Nicolas Ballarini, CeMSIIS, Medical University of Vienna
% nicoballarini@gmail.com
% http://statistics.msi.meduniwien.ac.at/
%
% Customized for UAH by:
% David F. Barrero, Departamento de Automática, UAH
%%%%%%%%%%%%%%%%%%%%%%%%%%%%%%%%%%%%%%%%%%%%%%%%%%%%%%%%%%%%%%%%%

\documentclass[10pt,compress]{beamer} % Change 10pt to make fonts of a different size
\mode<presentation>

\usepackage[spanish]{babel}
\usepackage{fontspec}
\usepackage{tikz}
\usepackage{etoolbox}
\usepackage{xcolor}
\usepackage{xstring}
\usepackage{multicol}
\usepackage{listings}
\usepackage{dirtree}
\usepackage[absolute,overlay]{textpos}
\usepackage{tikz}
\usetikzlibrary{matrix,chains,positioning,decorations.pathreplacing,arrows,shapes}

\usetheme{UAH}
\usecolortheme{UAH}
\setbeamertemplate{navigation symbols}{} 
\setbeamertemplate{caption}[numbered]

%%%%%%%%%%%%%%%%%%%%%%%%%%%%%%%%%%%%%%%%%%%%%%%%%%%%%%%%%%%%%%%%%
%% Presentation Info
\title[More Python for Videogames]{More Python for Videogames}
\author{\asignatura\\\carrera}
\institute{}
\date{Departamento de Automática}
%%%%%%%%%%%%%%%%%%%%%%%%%%%%%%%%%%%%%%%%%%%%%%%%%%%%%%%%%%%%%%%%%


%%%%%%%%%%%%%%%%%%%%%%%%%%%%%%%%%%%%%%%%%%%%%%%%%%%%%%%%%%%%%%%%%
%% Descomentar para habilitar barra de navegación superior
\setNavigation
%%%%%%%%%%%%%%%%%%%%%%%%%%%%%%%%%%%%%%%%%%%%%%%%%%%%%%%%%%%%%%%%%

%%%%%%%%%%%%%%%%%%%%%%%%%%%%%%%%%%%%%%%%%%%%%%%%%%%%%%%%%%%%%%%%%
%% Configuración de logotipos en portada
%% Opacidad de los logotipos
\newcommand{\opacidad}{1}
%% Descomentar para habilitar logotipo en pié de página de portada
%\renewcommand{\logoUno}{Images/isg.png}
%% Descomentar para habilitar logotipo en pié de página de portada
%\renewcommand{\logoDos}{Images/CCLogo.png}
%% Descomentar para habilitar logotipo en pié de página de portada
%\renewcommand{\logoTres}{Images/ALogo.png}
%% Descomentar para habilitar logotipo en pié de página de portada
%\renewcommand{\logoCuatro}{Images/ELogo.png}
%%%%%%%%%%%%%%%%%%%%%%%%%%%%%%%%%%%%%%%%%%%%%%%%%%%%%%%%%%%%%%%%%

%%%%%%%%%%%%%%%%%%%%%%%%%%%%%%%%%%%%%%%%%%%%%%%%%%%%%%%%%%%%%%%%%
%% FOOTLINE
%% Comment/Uncomment the following blocks to modify the footline
%% content in the body slides. 


%% Option A: Title and institute
\footlineA
%% Option B: Author and institute
%\footlineB
%% Option C: Title, Author and institute
%\footlineC
%%%%%%%%%%%%%%%%%%%%%%%%%%%%%%%%%%%%%%%%%%%%%%%%%%%%%%%%%%%%%%%%%

\begin{document}

%%%%%%%%%%%%%%%%%%%%%%%%%%%%%%%%%%%%%%%%%%%%%%%%%%%%%%%%%%%%%%%%%
% Use this block for a blue title slide with modified footline
{\titlepageBlue
    \begin{frame}
        \titlepage
    \end{frame}
}

\institute{\asignatura}

\begin{frame}[plain]{}
	\begin{block}{Objectives}
		\begin{enumerate}
		\item Being able to manipulate files in Python.
		\item Understand and apply Python serialization (\textit{pickles} and JSON).
		\item Being able to handle exceptions.
%		\item Introduce decorators
		\end{enumerate}
	\end{block}

	\begin{block}{Bibliography}
		\begin{itemize}
			\item The Python Tutorial. \textit{Section 7.2: Reading and writing files}. \href{https://docs.python.org/3/tutorial/inputoutput.html\#reading-and-writing-files}{(Link)}
			\item The Arcade Library. \textit{Sound}. \href{https://api.arcade.academy/en/3.3.3/programming\_guide/sound.html}{(Link)}
			\item Learn Arcade. \textit{Chapter 20: Sound}. \href{https://learn.arcade.academy/en/latest/chapters/20_sounds/sounds.html}{(Link)}
		\end{itemize}
	\end{block}
\end{frame}

{
\disableNavigation{white}
\begin{frame}[shrink]{Table of Contents}

 	\frametitle{Table of Contents}
  	\begin{multicols}{2}
  		\tableofcontents
    	\end{multicols}

 %\frametitle{Table of Contents}
 %\tableofcontents
  % You might wish to add the option [pausesections]
\end{frame}
}

\section{Path}
\subsection{Definition}
\begin{frame}[fragile]{Path}{Definition}
	\begin{columns}
 	   \column{0.35\textwidth}
	\dirtree{%
.1 root.
.2 home.
.3 rick.
.4 foo.py.
.4 music.mp3.
.3 morthy.
.5 $...$ .
.2 bin .
.2 $...$ .
}
 	   \column{0.65\textwidth}

    \begin{block}{Path}
	A string that identifies a file in a file system
    \end{block}
	Two types of paths:
		\begin{itemize}
			\item Absolute: Address from the root directory
			\item Relative: Address from the \alert{working directory}
		\end{itemize}
	Working directory: Folder where your program is running from
	\end{columns}

	\bigskip

	The path separator is operating system dependent
	\begin{itemize}
		\item Linux/macOS/Android: Uses forward slash: \texttt{/}
		\item Windows: Uses backslash: \texttt{$\backslash$}
	\end{itemize}
\end{frame}

\subsection{Paths in Linux}

\begin{frame}[fragile]{Path}{Paths in Linux}

	\begin{columns}
 	  \column{.40\textwidth}
	\dirtree{%
.1 /.
.2 home/.
.3 rick/.
.4 foo.py.
.4 music.mp3.
.3 morthy/.
.5 $...$ .
.2 bin/ .
.2 $...$ .
}
 	  \column{.60\textwidth}
		On Linux, the absolute path is: \\
\begin{verbatim}
path = '/home/rick/music.mp3'
\end{verbatim}
		If we are in \texttt{/home/}, the relative path is: \\
\begin{verbatim}
path = 'rick/music.mp3'
\end{verbatim}

	\end{columns}
\end{frame}

\subsection{Paths in Windows}

\begin{frame}[fragile]{Path}{Paths in Windows}

	\begin{columns}
 	  \column{.40\textwidth}
	\dirtree{%
.1 C:\textbackslash.
.2 Users\textbackslash.
.3 rick\textbackslash.
.4 foo.py.
.4 music.mp3.
.3 morthy\textbackslash.
.5 $...$ .
.2 $...$ .
}
 	  \column{.60\textwidth}
		On \textbf{Windows}, the absolute path is: \\
\begin{verbatim}
'C:\Users\rick\music.mp3'
\end{verbatim}
		If we are in \texttt{C:\textbackslash Users\textbackslash}, the relative path is: \\
\begin{verbatim}
'rick\music.mp3'
\end{verbatim}
		And it is represented in Python by:\\
\begin{verbatim}
path = 'C:\\Users\\rick\\music.mp3'
\end{verbatim}
		But by also using \textit{raw string}:
\begin{verbatim}
path = r'C:\Users\rick\music.mp3'
\end{verbatim}
	\end{columns}
\end{frame}

\begin{frame}{Path}{Portable code with the os module}
	
	The \texttt{os} module provides old fashioned tools to deal with paths
	\begin{itemize}
		\item \texttt{os.path.join("folder", "subfolder", "file.txt")}
		\item \texttt{os.sep}
	\end{itemize}

	\vspace{-0.5cm}
	\begin{columns}
 	  \column{.50\textwidth}
	\begin{exampleblock}{Recommended}
	\vspace{-0.2cm}
	\lstinputlisting[numbers=left]{code/os1.py}
	\vspace{-0.2cm}
	\end{exampleblock}

 	  \column{.50\textwidth}
	\begin{exampleblock}{Not recommended}
	\vspace{-0.2cm}
	\lstinputlisting[numbers=left]{code/os2.py}
	\vspace{-0.2cm}
	\end{exampleblock}

	\begin{alertblock}{Use with caution}
	\vspace{-0.2cm}
	\lstinputlisting[numbers=left]{code/os3.py}
	\vspace{-0.2cm}
	\end{alertblock}

	\end{columns}
	\bigskip
	Pathlib provides a modern (i.e. object-oriented) solution
\end{frame}

\subsection{The \texttt{\_\_file\_\_} variable}

\begin{frame}{Path}{The \texttt{\_\_file\_\_} variable}
	Python defines the variable \alert{\texttt{\_\_file\_\_}}
	\begin{itemize}
		\item Contains the absolute path to the file
		\item Not defined if there is no file
		\item Useful to locate our project location
	\end{itemize}
	\begin{textblock*}{3cm}(8cm, 4cm)
		\includegraphics[width=\linewidth]{figs/meme-path.jpg}
	\end{textblock*}
\end{frame}

\section{Reading and writing files}
\subsection{Introduction}
\begin{frame}[fragile]{Reading and writing files}{Introduction}
	\begin{block}{File operation overview}
	\begin{enumerate}
	\item Open the file in a specific mode
	\item Perform operations on the file (read/write, among others)
        \item Close the file
	\end{enumerate}
	\end{block}

	All file operations are performed through a \textit{file object}
		\begin{itemize}
		\item First: call the \texttt{open()} function
		\item It returns the file object
		\item Always close the file, even in the event of failure
		\end{itemize}
\end{frame}

\subsection{Opening files}
\begin{frame}[fragile]{Reading and writing files}{Opening files (I)}
	\begin{block}{\texttt{open()}}
	\vspace{-0.2cm}
\begin{verbatim}
open(path[, mode])
\end{verbatim}
	\vspace{-0.2cm}
	Return: An object file\\
	\vspace{-0.2cm}
	\begin{itemize}
	\item \textit{path}: Path string
	\item \textit{mode}: Characters describing how the file will be used
		\begin{itemize}
		\item \textit{r}: Read mode, \textit{w}: Write mode (overwrites file)% lectura por defecto. W: borra el contenido anterior, r+: lectura y escritura; a: append
        \item \textit{r+}: r/w mode, no truncation; \textit{w+}: r/w mode, truncation
        \item \textit{a}: Write, appending mode
		\item \textit{b}: Binary mode, text mode by default
		\end{itemize}
	\end{itemize}
	\end{block}

	\smallskip

	\begin{alertblock}{}
	Always, always, always close the file: \texttt{f.close()} (unless in a \texttt{with} clause)
	\end{alertblock}
\end{frame}

\begin{frame}[fragile]{Reading and writing files}{Opening files (II)}
	\begin{columns}
 	  \column{.50\textwidth}
	\begin{exampleblock}{Text mode}
	\vspace{-0.2cm}
	\lstinputlisting[numbers=left]{code/open1.py}
	\vspace{-0.2cm}
	\end{exampleblock}

 	  \column{.50\textwidth}
	\begin{exampleblock}{Binary mode}
	\vspace{-0.2cm}
	\lstinputlisting[numbers=left]{code/open2.py}
	\vspace{-0.2cm}
	\end{exampleblock}
	\end{columns}
\end{frame}

\subsection{Reading files}
\begin{frame}[fragile]{Reading and writing files}{Reading files (I)}
	\begin{block}{The \texttt{read()} method}
	\vspace{-0.2cm}
\begin{verbatim}
f.read([size])
\end{verbatim}

	\vspace{-0.2cm}
	Return: the specified number of bytes
	\begin{itemize}
	\item \textit{size}: The number of bytes to be read from the file. Default reads the whole file
	\end{itemize}
	\end{block}
\textbf{Option 1}: Read the entire file (\texttt{f.read()})
\begin{verbatim}
>>> f = open("/tmp/file", 'r')
>>> f.read()
'This is the entire file.\\n'
>>> f.read()
''
>>> f.close()
\end{verbatim}
	
\end{frame}

\begin{frame}[fragile]{Reading and writing files}{Reading files (II)}
	\textbf{Option 2}: Read a single line (\texttt{f.readline()})
\begin{verbatim}
>>> f = open("/tmp/file2", 'r')
>>> f.readline()
'This is the first line of the file.\n'
>>> f.readline()
'This is the second line of the file\n'
>>> f.readline()
''
>>> f.close()
\end{verbatim}

\end{frame}

\begin{frame}[fragile]{Reading and writing files}{Reading files (III)}
	\textbf{Option 3}: Read lines as list (\texttt{f.readlines()})
\begin{verbatim}
>>> f = open("/tmp/file2", 'r')
>>> f.readlines()
['This is the first line of the file.\n',
'This is the second line of the file\n']
>>> f.close()
\end{verbatim}

	\textbf{Option 4}: Read in a loop
\begin{verbatim}
f = open("/tmp/file2", 'r')
for line in f:
    print(line, end='')
f.close()
\end{verbatim}
\end{frame}

%\subsection{Reading files: Example}

\begin{frame}[fragile]{Reading and writing files}{Example}{}
	\begin{exampleblock}{Number of lines and characters in file \texttt{example.txt}}
	\vspace{-0.2cm}
	\lstinputlisting[numbers=left]{code/char_average_line.py}
	\vspace{-0.2cm}
	\end{exampleblock}
\end{frame}

\subsection{Writing files}

\begin{frame}[fragile]{Reading and writing files}{Writing files (I)}
	\begin{block}{The \texttt{write()} method}
	\vspace{-0.2cm}
\begin{verbatim}
f.write(string)
\end{verbatim}

	\vspace{-0.2cm}
	Return: Number of written bytes
	\begin{itemize}
	\item \textit{string}: String to write
	\end{itemize}
	\end{block}

	\textbf{Example 1}: Write a line
	 % si el archivo se crea con modo w+: se sobrescribe lo que hubiera al estar ya creado.
	 % no se lee luego nada porque el puntero está posicionado al final y no hay más.
	\begin{verbatim}
>>> f = open("/tmp/file", 'w+')
>>> f.write('This is a test\n')
15
>>> f.read()
'' 
>>> f.close()
\end{verbatim}

\end{frame}

\begin{frame}[fragile]{Reading and writing files}{Writing files (II)}
	\textbf{Example 2}: Write a number % se tiene que pasar a string el número
\begin{verbatim}
>>> f = open("/tmp/file", 'w+')
>>> f.write(str(42))
2
>>> f.close()
\end{verbatim}
\end{frame}


%\subsection{Writing files: Example}

\begin{frame}[fragile]{Reading and writing files}{Writing files: Example}
	\begin{exampleblock}{}
	\vspace{-0.2cm}
	\lstinputlisting[numbers=left]{code/write.py}
	\vspace{-0.2cm}
	\end{exampleblock}
\end{frame}


\subsection{Random access}
\begin{frame}[fragile]{Reading and writing files}{Random access (I)}

	\begin{tabular}{c|l}\hline
  	\sc Method & \sc Description  \\\hline
  	\texttt{f.tell()} & Returns the pointer's position \\
  	\texttt{f.seek(n)} & Moves the pointer to position \texttt{n} \\\hline
  	\end{tabular}

\begin{verbatim}
>>> f = open("/tmp/file", 'rb+')
>>> f.write(b'0123456789abcdef')
16
>>> f.seek(5)
5
>>> f.read(1)
b'5'
>>> f.close()
\end{verbatim}
\end{frame}

\begin{frame}[plain]{Reading and writing files}{Random access (II)}
	\begin{columns}
		\includegraphics[width=\linewidth]{figs/pointer.pdf}
	\end{columns}
\end{frame}

\subsection{With}

\begin{frame}[fragile]{Reading and writing files}{\texttt{With} (I)}
	\begin{columns}
	\column{.6\textwidth}
	\begin{block}{The \texttt{with} clause}
	\vspace{-0.2cm}
\begin{verbatim}
with open(path, mode) as file:
    ...
\end{verbatim}
	\end{block}

	\column{.4\textwidth}

	It simplifies file operations
	\begin{itemize}
	\item No need to close files
    \item Better exception handling
	\end{itemize}
    \end{columns}

    
	\begin{columns}
	\column{.45\textwidth}
	\begin{exampleblock}{}
	\begin{verbatim}
f = open('file')
print(f.read())
f.close()
\end{verbatim}
    \end{exampleblock}

	\column{.05\textwidth}
    $\Rightarrow$

	\column{.50\textwidth}
	\begin{exampleblock}{}
	\begin{verbatim}
with open('file') as file:
    print(file.read())
\end{verbatim}
    \end{exampleblock}

    \end{columns}
\end{frame}

\begin{frame}[fragile]{Reading and writing files}{With (II)}
	\begin{exampleblock}{Hello, world}
	\vspace{-0.2cm}
	\lstinputlisting[numbers=left]{code/hello-with.py}
	\vspace{-0.2cm}
	\end{exampleblock}
\end{frame}

\begin{frame}[fragile]{Reading and writing files}{With (III)}
	\begin{exampleblock}{Reading a line each time}
	\vspace{-0.2cm}
	\lstinputlisting[numbers=left]{code/cierreautomatico.py}
	\vspace{-0.2cm}
	\end{exampleblock}

	\begin{columns}
	\column{.40\textwidth}
    \centering $\Uparrow$
	\begin{exampleblock}{names.txt}
	\lstinputlisting[basicstyle=\scriptsize, numbers=left]{code/nombres.txt}
	\end{exampleblock}
	\column{.50\textwidth}
    \centering $\Downarrow$
	\begin{exampleblock}{Output}
	\lstinputlisting[basicstyle=\scriptsize]{code/output.txt}
	\end{exampleblock}
	\end{columns}
	
\end{frame}

\section{Pathlib}
\subsection{Introduction}
\begin{frame}{Pathlib}{Introduction}
	Pathlib is a module for working with paths
	\begin{itemize}
		\item Built-in module from Python 3.4
		\item Object-oriented
		\item Intuitive path operations
		\item Methods for common operations
		\item Supported by Arcade 3.x
		\item \href{https://docs.python.org/3/library/pathlib.html}{(Pathlib reference documentation)}
	\end{itemize}

	\begin{columns}
	\column{.40\textwidth}
		\begin{exampleblock}{os.path}
		\lstinputlisting[basicstyle=\scriptsize, numbers=left]{code/ospath.py}
		\end{exampleblock}
	\column{.50\textwidth}
		\begin{exampleblock}{pathlib}
		\lstinputlisting[basicstyle=\scriptsize, numbers=left]{code/pathlib1.py}
		\end{exampleblock}
	\end{columns}
\end{frame}

\subsection{Creating paths}
\begin{frame}{Pathlib}{Creatings paths}
	Basic Path
	\begin{itemize}
		\item[] \texttt{path = Path('folder/subfolder/file.txt')}
	\end{itemize}

	Using / operator
	\begin{itemize}
		\item[] \texttt{path = Path('folder') / 'subfolder' / 'file.txt'}
	\end{itemize}

	Current directory
	\begin{itemize}
		\item[] \texttt{path = Path.cwd()}
	\end{itemize}

	Home directory
	\begin{itemize}
		\item[] \texttt{path = Path.home()}
	\end{itemize}
\end{frame}

\subsection{Common operations}
\begin{frame}{Pathlib}{Common operations}
	\begin{columns}
	\column{.50\textwidth}
	Check if path exists
	\begin{itemize}
		\item[] \texttt{path.exists()}
	\end{itemize}

	Check if path is a file
	\begin{itemize}
		\item[] \texttt{path.is\_file()}
	\end{itemize}

	Check if path is a directory
	\begin{itemize}
		\item[] \texttt{path.is\_dir()}
	\end{itemize}

	\column{.50\textwidth}
	Create directory
	\begin{itemize}
		\item[] \texttt{path.mkdir()}
	\end{itemize}

	Write to file
	\begin{itemize}
		\item[] \texttt{path.write\_text()}
	\end{itemize}

	Read file content
	\begin{itemize}
		\item[] \texttt{path.read\_text()}
	\end{itemize}
	\end{columns}
\end{frame}


\subsection{Example}

\begin{frame}[fragile]{Reading and writing files}{Example}

	\vspace{-0.5cm}
	\begin{exampleblock}{}
	\vspace{-0.2cm}
	\lstinputlisting[numbers=left]{code/pathlib-example.py}
	\vspace{-0.2cm}
	\end{exampleblock}
\end{frame}


\section{Serialization}
\begin{frame}[fragile]{Serialization}{Introduction}
    What happens if we need to store complex data structures?
		\begin{itemize}
			\item Think about lists, dictionaries or even objects ...
		\end{itemize}
    What happens if we need to transmit complex data structures?

	\begin{columns}
	\column{.5\textwidth}
	\begin{block}{Serialization}
    Converting a data object into a sequence of bytes
	\end{block}
	\column{.5\textwidth}
	\begin{block}{Deserialization}
    Converting a sequence of bytes into a data object
	\end{block}

    \end{columns}

    \bigskip

    We can easily store and even transmit sequences of bytes ...
	\begin{itemize}
        \item ... and also reconstruct our original data
	\end{itemize}
	There are several serialization technologies: Pickles, JSON, XML, YAML, ...

\end{frame}

\subsection{The \texttt{pickle} module}

\begin{frame}[fragile]{Serialization}{The \texttt{pickle} module}
	 \includegraphics[width=0.8\linewidth]{figs/pickle.png} \\
	 \bigskip
	%\begin{columns}
 	%  \column{.50\textwidth}
		Given an object \texttt{x} and a file object \texttt{f} ...
		\begin{itemize}
			\item \texttt{pickle.dump(x, f)}: Serializes object x and writes it to file f
			\item \texttt{pickle.load(f)}: Reads and deserializes object from file f
			\item \texttt{x} may be a dictionary, list or even an object
		\end{itemize}
		Pickle uses a binary format
 	 % \column{.50\textwidth}
	%\end{columns}
\end{frame}

%\subsection{The \texttt{pickle} module: Examples}

\begin{frame}[plain]{Serialization}{The \texttt{pickle} module: Examples}


	\begin{columns}
 	  \column{.50\textwidth}
	\begin{exampleblock}{Save a list  to a file}
	\vspace{-0.2cm}
	\lstinputlisting[numbers=left]{code/pickle_example_list_dump.py}
	\vspace{-0.2cm}
	\end{exampleblock}

 	  \column{.50\textwidth}
	\begin{exampleblock}{Load a list from a file}
	\vspace{-0.2cm}
	\lstinputlisting[numbers=left]{code/pickle_example_list_load.py}
	\vspace{-0.2cm}
	\end{exampleblock}
	\end{columns}
	
	\bigskip
	\centering \includegraphics[width=0.25\linewidth]{figs/meme-pickle.jpg} \includegraphics[width=0.25\linewidth]{figs/meme-pickle.png} 
\end{frame}

\subsection{The JSON module}

\begin{frame}[fragile]{Serialization}{The \texttt{JSON} module}
	 \includegraphics[width=0.8\linewidth]{figs/json.png} \\
	 \bigskip
	%\begin{columns}
 	%  \column{.50\textwidth}
		Given an object \texttt{x} and a file object \texttt{f} ...
		\begin{itemize}
			\item \texttt{json.dump(x, f)}: Serializes object x and writes it to file f
			\item \texttt{json.load(f)}: Reads and deserializes object from file f
			\item More limited than pickles
		\end{itemize}
		JSON is a text format
 	 % \column{.50\textwidth}
	%\end{columns}
\end{frame}

\subsection{The JSON module: JSON format}

\begin{frame}[fragile]{Serialization}{The \texttt{JSON} module: JSON format}
	\begin{columns}
 	  \column{.40\textwidth}
		JSON: \textit{JavaScript Object Notation}
		\begin{itemize}
		\item Data format for hierarchical data
        	\item Created in 2001 for stateless client-server communication
	        \item Text-based
		\item Interoperable (pickles only for Python)
	        \item Complex data structures
		\end{itemize}
		%Tile Map Editor generates JSON files!

 	  \column{.60\textwidth}
        \begin{exampleblock}{filename.json}
		\begin{lstlisting}[basicstyle=\scriptsize]
{
  "firstName": "John",
  "isAlive": true,
  "age": 27,
  "address": {
    "streetAddress": "21 2nd Street",
    "city": "New York",
    "state": "NY",
  },
  "phoneNumbers": [ "111", "333" ]
}
\end{lstlisting}
        \end{exampleblock}
	\end{columns}
\end{frame}

%\subsection{The JSON module: Examples}

\begin{frame}[fragile]{Serialization}{The \texttt{JSON} module: Examples}
	\begin{exampleblock}{Save a list to a file}
	\vspace{-0.2cm}
	\lstinputlisting[numbers=left]{code/json_example_list_dump.py}
	\vspace{-0.2cm}
	\end{exampleblock}

	\begin{exampleblock}{Load a list from a file}
	\vspace{-0.2cm}
	\lstinputlisting[numbers=left]{code/json_example_list_load.py}
	\vspace{-0.2cm}
	\end{exampleblock}

	\vspace{-2cm}
	{\raggedright
	\hfill \includegraphics[width=0.25\linewidth]{figs/json-meme.jpg}
	}
\end{frame}

\section{Exceptions in I/O}
\subsection{Motivation}

\begin{frame}{Exceptions in I/O}{Motivation}
    Errors happen ... more often in I/O operations
	\begin{itemize}
	\item File does not exist
	\item No permission to read/write
	\item Disk full
	\item Incorrect type of file
    \end{itemize}
    $\Rightarrow$ We need tools to handle errors: \alert{Exceptions}
\end{frame}

\subsection{Common I/O exceptions}

\begin{frame}{Exceptions in I/O}{Common I/O exceptions}
    I/O operations may raise the following exceptions
    \begin{itemize}
	\item \texttt{FileNotFoundError}\\ File does not exist
	\item \texttt{PermissionError} \\No permission to read/write
	\item \texttt{IOError} \\Disk full and other I/O errors
	\item \texttt{IsADirectoryError} and \texttt{NotADirectoryError}\\ Incorrect type of file
    \end{itemize}
\end{frame}

\subsection{Examples}
\begin{frame}{Exceptions in I/O}{Examples (I)}
    \begin{columns}
 	   \column{\textwidth}
	\begin{exampleblock}{try-except}
	\vspace{-0.2cm}
		\lstinputlisting{code/try-example3.py}
	\vspace{-0.2cm}
	\end{exampleblock}
	\end{columns}
\end{frame}

\begin{frame}{Exceptions in I/O}{Examples (II)}
    \begin{columns}
 	   \column{\textwidth}
	\begin{exampleblock}{try-except statement}
	\vspace{-0.2cm}
		\lstinputlisting{code/try-example4.py}
	\vspace{-0.2cm}
	\end{exampleblock}
	\end{columns}
\end{frame}

\begin{frame}{Exceptions in I/O}{Examples (III)}
    \begin{columns}
 	   \column{\textwidth}
	\begin{exampleblock}{try-except statement}
	\vspace{-0.2cm}
		\lstinputlisting{code/try-example5.py}
	\vspace{-0.2cm}
	\end{exampleblock}
	\end{columns}
\end{frame}


%\begin{frame}{Exceptions}{Exception definition (II)}
%    \begin{columns}
% 	   \column{.50\textwidth}
%		\centering Call stack
%		\centering \includegraphics[width=0.8\linewidth]{figs/exceptions-callstack.png}

%  	\column{.50\textwidth}
%		\textit{Call stack}: Sequence of invoked methods
%	\end{columns}
%\end{frame}

%\begin{frame}{Exceptions}{Exception definition (III)}
%    \begin{columns}
% 	   \column{.60\textwidth}
%		\centering Exception handling
%		\includegraphics[width=0.8\linewidth]{figs/exceptions-errorOccurs.png}

%  	\column{.60\textwidth}
%		When an error happens ...
%		\begin{enumerate}
%		\item Code execution is stopped
%		\item An exception is thrown
%		\item The interpreter goes back in the call stack
%		\item When the interpreter finds an exception handler, it is executed
%		\end{enumerate}
%		The exception handler catches the exception, the program finishes otherwise
%	\end{columns}
%\end{frame}

%\begin{frame}{Exceptions}{Exception definition (IV)}
%	\lstinputlisting{code/stack.txt}
%\end{frame}

%\subsection{Handling exceptions}
%\begin{frame}{Exceptions}{Handling exceptions (I)}
%	Handling an exception requires a try-except statement
%	\begin{itemize}
%		\item \texttt{try}: Encloses the vulnerable code
%		\item \texttt{except}: Code that handles the exception (\textit{catch} in Java/C++)
%	\end{itemize}

%    \begin{columns}
% 	   \column{.60\textwidth}
%	\begin{block}{try-except statement}
%	\vspace{-0.2cm}
%		\lstinputlisting{code/try.py}
%	\vspace{-0.2cm}
%	\end{block}
%	\end{columns}
%\end{frame}

%\begin{frame}{Exceptions}{Handling exceptions (II)}
%    \begin{columns}
% 	   \column{.90\textwidth}
%	\begin{exampleblock}{try-except example}
%	\vspace{-0.2cm}
%		\lstinputlisting[numbers=left]{code/try-example.py}
%	\vspace{-0.2cm}
%	\end{exampleblock}
%	\end{columns}
%	\bigskip
%	The exception type contains the error
%\end{frame}

%\begin{frame}{Exceptions}{Handling exceptions (III)}
%	\vspace{-0.2cm}
%    \begin{columns}
% 	   \column{.90\textwidth}
%	\begin{exampleblock}{try-except example}
%	\vspace{-0.2cm}
%		\lstinputlisting[numbers=left]{code/try-example2.py}
%	\vspace{-0.2cm}
%	\end{exampleblock}
%	\end{columns}
%	\bigskip
%	New Python elements
%		\begin{itemize}
%		\item Raise
%        \item Exception as object
%		\end{itemize}
%\end{frame}

%\subsection{Clean-up actions}
%\begin{frame}{Exceptions}{Clean-up actions}
%	\vspace{-0.2cm}
%	Sometimes we need to execute code under all circumstances
%	\begin{itemize}
%		\item Typically clean-up actions: Close files, database connections, sockets, etc
%		\item The \textbf{finally} clause solves this problem
%	\end{itemize}

%	\vspace{-0.2cm}
%    \begin{columns}
% 	   \column{.60\textwidth}
%	\begin{exampleblock}{Example}
%	\vspace{-0.2cm}
%		\lstinputlisting[numbers=left]{code/try-finally.py}
%	\vspace{-0.2cm}
%	\end{exampleblock}

%	\end{columns}
%\end{frame}

%\section{Decorators}
%\begin{frame}{Decorators}

%	TODO
%	\vspace{-0.2cm}
%	Sometimes we need to execute code under all circumstances
%	\begin{itemize}
%		\item Typically clean-up actions: Close files, database connections, sockets, etc
%		\item The \textbf{finally} clause solves this problem
%	\end{itemize}

%	\vspace{-0.2cm}
%    \begin{columns}
% 	   \column{.60\textwidth}
%	\begin{exampleblock}{Example}
%	\vspace{-0.2cm}
%		\lstinputlisting[numbers=left]{code/try-finally.py}
%	\vspace{-0.2cm}
%	\end{exampleblock}

%	\end{columns}
%\end{frame}



\section{Sound in Arcade}

\subsection{Introduction}
\begin{frame}{Sound in Arcade}{Introduction}
	Two steps:
	\begin{enumerate}
		\item Load the sound 
		\item Play the sound 
	\end{enumerate}
	Two ways to provide a path
	\begin{itemize}
		\item String\\ \texttt{path = 'laser.wav'}
		\item Path\\ \texttt{path = Path('laser.wav')}
	\end{itemize}

\end{frame}

\subsection{Loading sounds}
\begin{frame}{Sound in Arcade}{Loading sounds}
	Arcade supports two APIs
	\begin{itemize}
		\item Functional API: \\ \texttt{laser\_sound = arcade.load\_sound("laser.wav")}
		\item Object oriented API: \\ \texttt{laser\_sound = Sound("laser.wav")}
		\item [] Both return a \texttt{Sound} object
	\end{itemize}
	Boolean \texttt{streaming} argument
	\begin{itemize}
		\item True: Streams from disk. Long files
		\item False: Loads the whole file. Short files
	\end{itemize}
\end{frame}

\subsection{Playing sounds}
\begin{frame}{Sound in Arcade}{Playing sounds}
	Two ways to play a sound:
	\begin{itemize}
		\item The \texttt{arcade.play\_sound()} function\\ \texttt{arcade.play\_sound(laser\_sound)}
		\item The \texttt{play()} method (object oriented)
	\end{itemize}

	Both return a \texttt{Player} object
	\begin{itemize}
		\item It controls the playback
	\end{itemize}

\end{frame}

\subsection{Built-in sounds}
\begin{frame}{Sound in Arcade}{Built-in sounds}
    \begin{columns}
 	   \column{0.4\textwidth}
	Arcade comes with a collection of built-in resources
	\begin{itemize}
		\item Sounds, music, sprites, ...
		\item Good for testing
		\item \href{https://api.arcade.academy/en/3.3.3/api\_docs/resources.html}{(Link)}
	\end{itemize}
	    \begin{block}{Built-in resources}
	 \centering \texttt{":resources:<path>"}
	    \end{block}

 	   \column{0.6\textwidth}
	\begin{exampleblock}{Terminal}
	\vspace{-0.2cm}
		\lstinputlisting{code/sound.py}
	\vspace{-0.2cm}
	\end{exampleblock}
	\centering $\Uparrow$ Does not work as script!
    \end{columns}
\end{frame}

\subsection{Example}
\begin{frame}{Sound in Arcade}{Example}
	\vspace{-0.5cm}
    \begin{columns}
 	   \column{\textwidth}
	\begin{exampleblock}{}
	\vspace{-0.2cm}
		\lstinputlisting[numbers=left]{code/music.py}
	\vspace{-0.2cm}
	\end{exampleblock}
    \end{columns}

\end{frame}




%\subsection{Arcade sound API}
%\begin{frame}{Sound in Arcade}{Sound file formats}
%	TODO
%\end{frame}
\end{document}
