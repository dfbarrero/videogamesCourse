%%%%%%%%%%%%%%%%%%%%%%%%%%%%%%%%%%%%%%%%%%%%%%%%%%%%%%%%%%%%%%%%%
% MUW Presentation
% LaTeX Template
% Version 1.0 (27/12/2016)
%
% License:
% CC BY-NC-SA 4.0 (http://creativecommons.org/licenses/by-nc-sa/3.0/)
%
% Created by:
% Nicolas Ballarini, CeMSIIS, Medical University of Vienna
% nicoballarini@gmail.com
% http://statistics.msi.meduniwien.ac.at/
%
% Customized for UAH by:
% David F. Barrero, Departamento de Automática, UAH
%%%%%%%%%%%%%%%%%%%%%%%%%%%%%%%%%%%%%%%%%%%%%%%%%%%%%%%%%%%%%%%%%

\documentclass[10pt,compress]{beamer} % Change 10pt to make fonts of a different size
\mode<presentation>

\usepackage[spanish]{babel}
\usepackage{fontspec}
\usepackage{tikz}
\usepackage{etoolbox}
\usepackage{xcolor}
\usepackage{xstring}
\usepackage{listings}

% Introduced by David
\usepackage{eurosym}

\usetheme{UAH}
\usecolortheme{UAH}
\setbeamertemplate{navigation symbols}{} 
\setbeamertemplate{caption}[numbered]

%%%%%%%%%%%%%%%%%%%%%%%%%%%%%%%%%%%%%%%%%%%%%%%%%%%%%%%%%%%%%%%%%
%% Presentation Info
\title[Introduction to videogames]{Introduction to videogames}
\author{\asignatura\\\carrera}
\institute{}
\date{Departamento de Automática}
%%%%%%%%%%%%%%%%%%%%%%%%%%%%%%%%%%%%%%%%%%%%%%%%%%%%%%%%%%%%%%%%%


%%%%%%%%%%%%%%%%%%%%%%%%%%%%%%%%%%%%%%%%%%%%%%%%%%%%%%%%%%%%%%%%%
%% Descomentar para habilitar barra de navegación superior
\setNavigation
%%%%%%%%%%%%%%%%%%%%%%%%%%%%%%%%%%%%%%%%%%%%%%%%%%%%%%%%%%%%%%%%%

%%%%%%%%%%%%%%%%%%%%%%%%%%%%%%%%%%%%%%%%%%%%%%%%%%%%%%%%%%%%%%%%%
%% Configuración de logotipos en portada
%% Opacidad de los logotipos
\newcommand{\opacidad}{1}
%% Descomentar para habilitar logotipo en pié de página de portada
\renewcommand{\logoUno}{Images/isg.png}
%% Descomentar para habilitar logotipo en pié de página de portada
%\renewcommand{\logoDos}{Images/CCLogo.png}
%% Descomentar para habilitar logotipo en pié de página de portada
%\renewcommand{\logoTres}{Images/ALogo.png}
%% Descomentar para habilitar logotipo en pié de página de portada
%\renewcommand{\logoCuatro}{Images/ELogo.png}
%%%%%%%%%%%%%%%%%%%%%%%%%%%%%%%%%%%%%%%%%%%%%%%%%%%%%%%%%%%%%%%%%

%%%%%%%%%%%%%%%%%%%%%%%%%%%%%%%%%%%%%%%%%%%%%%%%%%%%%%%%%%%%%%%%%
%% FOOTLINE
%% Comment/Uncomment the following blocks to modify the footline
%% content in the body slides. 


%% Option A: Title and institute
\footlineA
%% Option B: Author and institute
%\footlineB
%% Option C: Title, Author and institute
%\footlineC
%%%%%%%%%%%%%%%%%%%%%%%%%%%%%%%%%%%%%%%%%%%%%%%%%%%%%%%%%%%%%%%%%

\begin{document}

%%%%%%%%%%%%%%%%%%%%%%%%%%%%%%%%%%%%%%%%%%%%%%%%%%%%%%%%%%%%%%%%%
% Use this block for a blue title slide with modified footline
{\titlepageBlue
    \begin{frame}
        \titlepage
    \end{frame}
}

\institute{\asignatura}

\begin{frame}[plain]{}
   \begin{block}{Objectives}
   \begin{itemize}
   		\item Contextualize game development
		\item Introduce basic vocabulary
	\end{itemize}
	\end{block}

   \begin{block}{Bibliography}
      \begin{enumerate}
          \item  \textit{Desarrollo de Videojuegos, Arquitectura del Motor de Videojuegos}. UCLM.
      \end{enumerate} 
   \end{block}
\end{frame}


{
\disableNavigation{white}
\begin{frame}[shrink]{Table of Contents}
 \frametitle{Table of Contents}
 \tableofcontents
  % You might wish to add the option [pausesections]
\end{frame}
}

\section{Introduction}

\section[Motivation]{Motivation}
\begin{frame}{Motivation}
	Why videogames?
	\begin{itemize}
		\item They involve all the Computer Science disciplines
		\item Exciting problems from an intelectual perspective
		\item Benchmark for AI
		\item Career opportunities
		\item They are fun!
  	\end{itemize}
\end{frame}

\section[Definition]{Definition}
\begin{frame}{Definition (I)}
    \begin{columns}
 	   \column{.90\textwidth}
	   \begin{block}{Vallejo}
	   A videogame is a \textit{graphical application} in \textit{real-time} with an \textit{interaction} between the user and the game
	   \end{block}
	\end{columns}
	\bigskip
	Real-time: \textbf{In this context}, it means the need of generating a frame rate\\
	Interaction: Joystick, keyboard, mouse, body, ...
\end{frame}

\begin{frame}{Definition (II)}
	Alternative definitions:

  	\begin{itemize}
	\item A \textit{play activity} with \textit{rules} that involves \textit{conflict} (I. Scheiber)
	\item A game has ``ends and means'': an \textit{objective}, an \textit{outcome}, and a \textit{set of rules} to get there (D. Parlett)
	\item A game is an activity involving \textit{player decisions}, seeking \textit{objectives} within a ``limiting context'' (i.e. \textit{rules}) (C. Abt)
	%\item System in which players engage in an artificial \textit{conflict}, defined by \textit{rules}, that results in a quantifiable \textit{outcome} (E. Zimmerman)
	\end{itemize}
	Highly recommended reading: Raph Koster. \href{http://www.theoryoffun.com/theoryoffun.pdf}{A Theory of Fun}. O'Really, 2nd edition. 2014.
\end{frame}

\begin{frame}{Definition (III)}
	\begin{center}
	A personal perspective
	\end{center}

	\begin{center}
	\huge Videogame = \textcolor{blue}{Video} + \alert{game}
	\end{center}

\end{frame}

\begin{frame}{Definition (IV)}
    \begin{columns}
 	   \column{.70\textwidth}
		Elements to take into account

 	 	\begin{itemize}
		\item Story (Characters, goals, dialogs, etc)
		\item Graphics (3D models, animations, videos, etc)
		\item Sound (Music, sound effects, voice, etc)
		\item Logic (Game rules, programming, etc)
		\item Interface (\alert{HUD}, user interface, etc)
		\item \alert{Gameplay} and playability
		\end{itemize}

 	   \column{.40\textwidth}
		\centering\includegraphics[width=0.7\linewidth]{figs/ProtectronCA2}\\
	\end{columns}
\end{frame}

\section{Videogames development}
\begin{frame}{Videogames development (I)}
	Topics involved in videogames development
	\begin{itemize}
		\item Personal computers
		\item Microprocessors development
		\item Peripherals (specific for videogames)
		\item 3D technology
		\item Internet
		\item Videogames engine development
		\item Physics engines
		\item Graphical engines
		\item Software engineering
		\item Artificial Intelligence (AI)
	\end{itemize}
\end{frame}

\begin{frame}{Videogames development (II)}
	Recent elements involved in videogames development:
	\begin{itemize}
		\item Human-machine interfaces
		\item Social networks
		\item Mobile technologies
		\item Tablets
	\end{itemize}
\end{frame}

\section[Industry]{Industry}
\begin{frame}{Industry (I)}
	Industry involves
	\begin{itemize}
		\item Development, distribution, marketing and sales
		\item This involves software and hardware as well
	\end{itemize}
	Videogames generates more business than pictures and music
	\begin{itemize}
		\item $57.6$ billion euros in 2009, $91$ in 2016
		\begin{itemize}
		\item $41$ mobile gaming, $34$ PCs, $19$ free-to-play
		\end{itemize}
		\item Average videogame cost: 7.4 - 9.7M \euro
			\begin{itemize}
			\item Consolited franchises
			\end{itemize}
	\end{itemize}
\end{frame}

\begin{frame}{Industry (II)}
	\begin{itemize}
		\item PCs decrease as consoles increase sales
		\item From mid 80's consoles are the main platform
		\item Best revenues are in software
		\begin{itemize}
		\item Hardware sold at a loss
		\end{itemize}
	\end{itemize}

	\begin{center}
	\includegraphics[width=0.3\linewidth]{figs/ps3.jpeg}
	\includegraphics[width=0.15\linewidth]{figs/xbox.jpeg}
	\includegraphics[width=0.2\linewidth]{figs/gameboy}
	\end{center}
\end{frame}

\section{History}
\begin{frame}{Overview of Videogames}{History}
	We can distinguish the following chronology
	   	\begin{enumerate}
		\item Videogames pre-history: Analogic hardware
		\item 80's: 8 bit (Spectrum, Amstrad, ...)
		\item 90's: 16 bit (Amiga, Atari, Game Boy, ...)
		\item 2000 to now: High performance hardware
	  	\end{enumerate}

	Check out these videos:
    	\begin{itemize}
		\item \href{http://www.youtube.com/watch?v=vRvclH7tRvk}{(Video past)}
		\item \href{http://www.youtube.com/watch?v=9t3TCecduQc}{(Video future)}
		\item \href{http://www.rtve.es/noticias/historia-videoconsolas/}{(Video suggested)}
    	\end{itemize}
\end{frame}



\end{document}
